%----------------------------------------------------------------------------
\chapter{\bevezetes}
%----------------------------------------------------------------------------
In this section the concepts of this paper are overviewed. 

\section{Runtime verification of CPSs}
Cyber-physical systems (CPS) \cite{Krupitzer2015,Nielsen2015,Sztipanovits2012}  are systems built of physical and computing units cooperating with each other.
Cyber-physical systems are used in various contexts. 
Factories, logistics and energy systems can utilize computers to improve quality, provide monitoring features, and ensure safety for them.

%As cyber-physical systems are widely used in critical environments, ensuring their safe operation is inevitable: their fault can lead to enormous financial loss, environmental damages, or even death ofaltoug people.
Design-time checking may solve some of the problems, but the physical component of the CPS makes them less feasible to formal model checking, as the state space of the physical worlds is practically infinite and is less expectable.
Another method is runtime verification \cite{fasecikk}: although we cannot prepare the system to avoid all of the problems, we can observe its state and detect if some errors occur and execute safety protocols to avoid harm. 


In this thesis a framework is presented capable of generating monitoring code, thus helping the runtime verification of cyber-physical systems. 
The framework is created to help developing such systems using model-driven development, where models are the primary artifacts in the development process.
Monitoring components do not have to be programmed manually but can be generated by our framework based on the domain models and error conditions given as graph patterns, which are defined by the developers of the system.

\section{Contributions and structure of the thesis}

I use the pronoun we to refer to my supervisors and me. 
The following points are my contribution to the topic:
\begin{itemize}
	\item I extended the local search based algorithm to work on distributed models on multiple computers,
	\item I implemented 
	\item I evaluated the framework.
\end{itemize}
	
	


In Chapter 2, the motivating example is presented, which will be useful to show how the concepts of this thesis are materialized in practice.
In Chapter 3, I show the technologies we used in the framework.
In Chapter 4, preliminaries are expounded to clarify the used terms and concepts and give some background knowledge needed to understand the framework.
Chapter 5 is an overview of the framework, its structure, and the approach it uses.
Chapter 6 is about domain modeling the system in Eclipse Modeling Framework.
Chapter 7 is about how graph queries can be defined using Viatra Query Language. It also shows some examples from the framework for the easier understanding of graph queries.
In Chapter 8 shows how we generate the model handling code from the domain model of the system.
In Chapter 9 shows how graph pattern is compiled into monitoring code and what other artifacts are generated for 
Chapter 10 shows how the generated artifacts work at runtime. 
In Chapter 11 we measure and evaluate the framework's performance.

\todo{Marci komment: ezt a végére hagyni és
runtime models and updates
graph based approach that are well-known in design time tools
we used theese for monitoring	
}

\todo{Itt valahogyutalnom kell arra, hogy mennyi indent csináltam, mert a sok passzívtól nem jön át?}