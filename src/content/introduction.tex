%----------------------------------------------------------------------------
\chapter{\bevezetes}
%----------------------------------------------------------------------------

Cyber-physial systems are used in various contexts. Factories, logistics and energy systems can utilize computers to improve quality, provide monitoring features, and ensure safety for them. In this thesis a framework is presented capable of generating monitoring code, thus helping the developement of cyber-physical systems. 

We use model-driven developement paradigm, meaning that models are the primary artifacts in the developement process for the framework. Monitoring components are not programmed manually, but generated by our framework based on the models defined by the developers of the system, who use the framework. 

In Chapter 2, a motivating example is presented to show the used approach on a concrete example. In Chapter 3, preliminaries are expound to clarify the used terms and concepts. Chapter 4 is an overview for the approach that the framework uses. In Chapter 5 the design time details are provided to show, how the system provides artifacts for monitoring. Chapter 6 shows how those generated artifacts work at runtime. In Chapter 7 we measure and evaluate the framework.