\chapter{Motivating example}

\todo{Újracsinálni ezt az egészet, ez így egy okádék}

We show the concepts of the framework using an example based on MoDeS3~\cite{modes3} demonstrator system. 

MoDeS3 (Model-based Demonstrator for Smart and Safe Systems) is a demonstrator system used for presenting the development and the capabilities of model-driven fault-tolerant systems.
It works on a model railway system. In this system, we have to ensure the correct operation of a railway system, eg.\ trains must not hit each other and derailments must be avoided. In a real railway system, we must avoid failure because it may cause serious damage, financial loss, or even death.

We use information from sensors perceiving the system's physical state to deduce if the state of the system is incorrect and/or dangerous. This is done by creating a structural model for the system and updating it based on sensor data coming from sensors. This structural model is called the live model. This model is graph based and we use graph pattern matching to find model parts indicating errors in the physical system. Graph pattern describes a subgraph satisfying given constraints, such as edges between two nodes must exist, given attributes of nodes must be a specific value or satisfy an expression, or another graph pattern must or must not match to a subset of nodes.

Monitoring goals for this system are defined as graph patterns. The usage and efficiency of the developed framework are demonstrated by defining these goals, compiling them to monitoring codes and running them on variously distributed models.

 



% célok és feladatok, amiket demonstrálni szeretnénk, monitoring goalok kiértékelése.S