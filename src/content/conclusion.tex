
\chapter{Conclusions}
\label{sec:conclusion}

In this paper we introduced a novel runtime verification system for distributed CPS systems. The approach is based on the VIATRA-Query graph pattern language, which is used to define analysis properties. Graph models are used to define the structural information captured from the system. A code generator generates the data structures and the code to evaluate the queries on the model at runtime. The model is continuously updated with the information received from the environment and fast query execution provides runtime verification. 
The framework combines various technologies from the model-driven and also the CPS domain into a comprehensive approach. A case-study and measurements are used for evaluation purposes. 
The initial measurements show that the approach is capable of analyzing real life systems and the framework can provide fast results even running in microcomputers with low resources.

In the future we plan to continue this line of research. We hope to extend the model handling to provide more adaptability and new strategies have to be added to efficiently handle highly dynamic systems. In addition, we plan to extend the approach to handle temporal properties.
So there is much work left and there are several points where we can further improve the framework. 